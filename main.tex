\documentclass[12pt,reqno]{amsart}

\usepackage{amssymb,amsfonts}
\usepackage[a4paper, top=3cm, left=3cm, right=4cm, bottom=5cm]{geometry}
\setlength{\parskip}{0pt}
\usepackage[ngerman]{datetime}

\usepackage{amsthm}
\usepackage{amsmath}
\usepackage{amssymb}
\usepackage[colorlinks = true, linkcolor = black, citecolor = black, final]{hyperref}
\usepackage{graphicx}
\usepackage{todonotes} 
\usepackage{multicol}


\newcommand{\ds}{\displaystyle}
\DeclareMathOperator{\sech}{sech}


\setlength{\parindent}{0in}

\pagestyle{empty}

\begin{document}

\thispagestyle{empty}

{\scshape \today} \hfill {\scshape \large Exposè} \hfill {\scshape Bachelor These}
 
\smallskip

\hrule

\section*{ Fragestellung/Aufgabenstellung}

\bigskip

\begin{itemize}
\item[(15.Juni)] Die Automatisierung des Knowledge Managements der fachlichen Software - Architektur in einem X-Team mit agiler Organisationsstruktur
\item[(22.Juni)]Untersuchung der Kopplung in einer Aws-Ressourcen Microservice-Landschaft mit Hilfe einer Graph-Datenbank
\item Untersuchung der Kopplung in einer Aws-Ressourcen Microservice-Landschaft 
\bigskip
\item[(30.Juni)] Untersuchung der Kopplung in einer Aws-Ressourcen Microservice- Landschaft mit agiler Organisationsstruktur
\item[(11.Juli)] Ist schwerfällige Kommunikation in einer agilen Organisationsstruktur ein Symptom für zu hoher Kopplung in einer Microservice Landschaft.
\textbf{\item[(12.Juli)] Schwerfällige Kommunikation als Symptom für zu hohe Kopplung in einer Microservice Landschaft.}
\item [(13.Juli)] Team Kommunikation als Metrik in einem Sotware- Projekt.
\end{itemize}

\bigskip


\section{ Einleitung - Problem}

\bigskip
\section*{Ausgangslage Motivation Problemstellung}
Das Projekt Deep-Sea, ist ein Software Projekt mit einer Größe von 20 Teams (a 15 Personen) in einer agilen Organisationsstruktur.\todo{Beschreiben des Projekts ?}[...]. Die Teams sollen die monolithische Software Landschaft des Unternehmens durch eine modulare Micro- Service (MS) Landschaft ablösen, um das Ziel einer isolierten Softwareentwicklung zu gewährleisten. Doch sind Architektur-Entscheidungen auch hier maßgeblich verantwortlich für die Wartbarkeit und  \todo{besser Beschreibung} [...].
Eine Herausforderung in einem MS System ist die Kopplung der Systeme untereinander, also X-Team weite Abhängigkeiten zwischen den Domains so klein wie möglich zu halten. Hier für wird eine eine klare Trennung der Kontext-Grenzen vorausgesetzt. Werden die fachlichen Komponenten zu unsauber geschnitten entsteht im Laufe der Entwicklung ein zu hoher Grad der Abhängigkeit, dies macht die Kommunikation zäh und langwierig. Teams untereinander merken nach der Zeit einen höheren Kommunikationsaufwand und auf enterprise Ebene kommt die Schwerfälligkeit erst an, wenn das Projekt auf rot steht. Was ein Problem für die Wirtschaflichkeit des Unternehmens darstellt.

\bigskip

\section*{Zielformulierung}

\bigskip
\textbf{Die Gruppen Kommunikation ist eine Metrik im Projekt die nicht zu vernachlässigen. Doch stellt sich die Frage in wie weit die Symptome der hohen Kommunikation abzubilden sind auf die Kopplung der Software Komponenten. }

\bigskip
Die Kommunikation der Teams wird in dem Microsoft Tool, MS-Teams durchgeführt.  Die  Vernetzung/Kommunikation der Teams kann so repräsentativ abgebildet werden \todo{ Was kann ich da anlaysieren den grad der Kommunikation durch zählen der eingehenden Kanten in einem Graph} [...]. 
Um die Kommunikation auf die Service Landschaft abzubilden muss auch in der Komponenten-Architektur die Abhängigkeiten  abgebildet werden, also Hardware Updates müssen festgehalten und Deepsea weit zusammen getragen werden.


\bigskip

\section*{ Methodische Herangehensweise}

\bigskip

Im Zeitrahmen meiner Bachelor Arbeit möchte ich die Updates der Komponenten \\ wöchentlich zusammen tragen und diese dann mit der Team-Kommunikation in kritischen Kopplungs-Bereichen vergleichen. 
Das Setup soll helfen Entscheidungen zu treffen, und Unstimmigkeiten in der Architektur zu refactorieren.


\bigskip

\section{Fragen}

\medskip

\begin{enumerate}

\item Welche Daten?
\item Woher die Daten? - (AWS \& MS-TEAMS)
\item 


\end{enumerate}


\end{document}